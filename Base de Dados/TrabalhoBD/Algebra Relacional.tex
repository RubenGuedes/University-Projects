\documentclass[12pt,a4paper]{article}
\begin{document}
\title{Trabalho Base de Dados}
\author{Sarah Luz n 38116 \\ Ruben Peixoto n 37514}
\date{}
\maketitle
\section*{Algebra Relacional}
\subsection*{Fundamentais}
%Pergunta 1
\textit{Pergunta 1}\\
O nome das pessoas que vivem em Portimao.\\
\begin{center}
$\pi$\begin{tiny}
nome
\end{tiny}$(\sigma$ \begin{tiny}
(nome,endereco = "Portimao")
\end{tiny}$(pessoa))$
\end{center} 
%Pergunta 2
\textit{Pergunta 2}\\
Encontrar o niv para as pessoas que compraram o carro antes de 2001.\\
\begin{center}
$\pi $\begin{tiny}
niv
\end{tiny}$ - \pi$ \begin{tiny}
niv
\end{tiny}$(\sigma$\begin{tiny}
(ano $\langle$ 2001)
\end{tiny}$(carro))$
\end{center}

\subsection*{Adicionais}
%Pergunta 3
\textit{Pergunta 3}\\
Nome das pessoas que tem um carro Toyota.\\
\begin{center}
temp1$\longleftarrow \pi$\begin{tiny}
niv
\end{tiny}$(\sigma$ \begin{tiny}
(niv, modelo = 'Toyota')
\end{tiny}$(carro))$
\end{center}
\begin{center}
temp2$\longleftarrow$temp1 $\bowtie$ tem
\end{center}
\begin{center}
$\pi$ \begin{tiny}
nome
\end{tiny}(temp2 $\bowtie$ pessoa)
\end{center}
%Pergunta 4
\textit{Pergunta 4}\\
O id condutor das pessoas que vivem em Portimao e que tiveram um acidente cujo o custo seja superior a 300 euros.\\
\begin{center}
temp1$\longleftarrow \pi$\begin{tiny}
id condutor
\end{tiny}$(\sigma $ \begin{tiny}
(id condutor, valor danos $\rangle$ 300)
\end{tiny}(participacao))
\end{center}
\begin{center}
temp2$\longleftarrow \pi$\begin{tiny}
id condutor
\end{tiny}$(\sigma $ \begin{tiny}
(id condutor, endereco = "Portimao")
\end{tiny}(pessoa))
\end{center}
\begin{center}
temp1 $\bigwedge$ temp2
\end{center}
\subsection*{Estendidas}
\textit{Pergunta 5}\\
Quantos carros da marca Toyota existem no ano 2017
\begin{center}
$\sigma$ \begin{tiny}
count(niv)
\end{tiny}$(\sigma$ \begin{tiny}
(modelo = "Toyota" $\bigwedge$ ano = 2017)
\end{tiny} (carro))
\end{center}
\textit{Pergunta 6}\\
Acrescentar 50 euros aos valores dos danos a todos os proprietarios que viveram acidentes em Lisboa.
\begin{center}
temp1$\longleftarrow \pi$\begin{tiny}
numero de relatorio
\end{tiny}$(\delta $ \begin{tiny}
(local = "Lisboa")
\end{tiny}$(acidente))$
\end{center}
\begin{center}
$\pi$ \begin{tiny}
(valor danos + 50)
\end{tiny}$(\delta$ \begin{tiny}
numero relatorio.temp1 = numero relatorio.participacao
\end{tiny}$(temp1,participacao)$
\end{center}
\end{document}


